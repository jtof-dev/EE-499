\documentclass[12pt, journal, onecolumn]{IEEEtran}
\IEEEoverridecommandlockouts{}

\usepackage[T1]{fontenc}
\usepackage{newtxtext}
\usepackage{newtxmath}

\usepackage{amsmath,amsfonts}
\interdisplaylinepenalty=2500
\usepackage{algorithmic}

\usepackage{cite}
\usepackage{graphicx}
\usepackage{textcomp}
\usepackage{xcolor}
\usepackage{url}
\usepackage[margin=1in]{geometry}

\usepackage[hidelinks]{hyperref}

\begin{document}

\title{Placeholder Title}

\author{Andy Babcock \\
    \textit{Steve Sanghi College of Engineering, Northern Arizona University} \\
    Flagstaff, AZ, USA \\
    aab726@nau.edu
}

\maketitle

\begin{abstract}
    I'm not sure if I want to include this, but here it is.
\end{abstract}

\section{Introduction}
Smart rings are a newly emerged wearable category, generally serving as a more minimalist alternative to smart watches. The Oura Ring was one of the first smart rings to come to market with a focus on health tracking, rather than the previous NFC-enabled smart rings~\cite{OuraWiki}. Since Oura’s launch in 2015, they have remained a constant in the smart ring space, even as competitors launched their own smart rings. In the past decade, the Oura Ring’s technology has been refined and homogenized across brands: sleep, activity, and general health are recorded by three to five sensors, and that data is interpreted by a companion phone app. The physical specifications have also been refined, with most smart rings being 2.5-5 grams, usually made with titanium, and with multiple days of battery life. Since current smart rings have similar physical specifications, their companies have to find something else to specialize in to help them stand out from the rest.

\section{Market}
Thanks to Oura’s relative longevity, their specialty is being the most well-rounded smart ring. Regardless of their ring’s comparative performance, they provide a consistently reliable product alongside a polished, useful phone app. One Oura competitor, the Samsung Galaxy Ring, differentiates itself by including tight software integration with their phones, most notably through finger tracking~\cite{SamsungWeb}. The Ultrahuman Ring AIR is another contender with added emphasis on how to decrease stress and improve general health through tips on their app~\cite{UltrahumanWeb}. A last competitor is the RingConn Gen 2 Air, which positions itself as a budget alternative that does not compromise on data accuracy~\cite{RingConnWeb}. All four smart ring companies market themselves towards the health-conscious consumer who does not want a smartwatch, and each carves out a small niche within that market. This more minimalist market that Oura occupies functions as a middle ground between the more feature-complete smartwatches and simpler, 'dumb' smart wearables that focus on doing one thing well. The Pebble Index 01 is Core Devices' first foray into the smart ring market, pulling from thir previous experience in making 'dumb' smartwatches. The Index 01 serves as external memory for the wearer's brain by allowing the user to record voice notes and add to to-do lists~\cite{PebbleWeb}. This smart ring similarly is marketed towards people who want a minimalist smartwatch alternative, but focuses more on being simple technology that keeps the user off their phone, rather than distracted. The Index 01 also has a companion app, which interprets the voice memo and creates notes or reminders accordingly. However, this app is designed to primarily work in the background, as opposed to health-focused smart ring apps which maximize the health data they can give the user.

\section{Sensors}
The Oura Ring 4 uses infrared LEDs, a digital temperature sensor, and an accelerometer to track health data~\cite{OuraWeb}. The infrared LEDs measure blood oxygen levels, heart rate, and heart rate variability, the digital temperature sensor tracks temperature trends and variation, and the accelerometer tracks movement and activity. Using the readings from these three sensors, Oura creates a full picture of the user's health, with their mobile app tracking user readiness, sleep, activity, daytime stress and resilience, heart health, and women's health~\cite{OuraWeb}. Oura's competitors generally use similar sensors, with some exceptions. The Ultrahuman AIR uses a 6-axis motion sensor instead of an accelerometer~\cite{UltrahumanWeb}, and the Samsung Galaxy Ring uses an optical bio-signal sensor rather than Oura's digital temperature sensor~\cite{SamsungWeb}.

Oura's accelerometer measures accelerations relative to the three Cartesian coordinate axis~\cite{AccelerometerVsGyro}. This measurement allows for tracking the changes in acceleration of a point, using the sensor itself as reference. Thus, an accelerometer cannot track its own position or rotation. A 6-axis motion sensor combines the accelerometer with a gyroscope which tracks changes in orientation around a reference axis, allowing for position and rotation tracking.

\bibliographystyle{IEEEtran}
\bibliography{references}

\end{document}
