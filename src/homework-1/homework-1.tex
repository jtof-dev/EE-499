\documentclass[journal, onecolumn]{IEEEtran}
\IEEEoverridecommandlockouts{}

\usepackage[T1]{fontenc}
\usepackage{newtxtext}
\usepackage{newtxmath}

\usepackage{amsmath,amsfonts}
\interdisplaylinepenalty=2500
\usepackage{algorithmic}

\usepackage{cite}
\usepackage{graphicx}
\usepackage{textcomp}
\usepackage{xcolor}
\usepackage{url}
\usepackage[margin=1in]{geometry}
\usepackage{booktabs}

\usepackage[hidelinks]{hyperref}

\begin{document}

\title{A Brief Analysis of the Oura Ring 4}

\author{Andy Babcock \\
    \textit{Steve Sanghi College of Engineering, Northern Arizona University} \\
    Flagstaff, AZ, USA \\
    aab726@nau.edu
}

\maketitle

\section{Introduction}
Smartrings are a newly emerged wearable category, generally serving as a minimalist alternative to smartwatches. The Oura Ring was one of the first smartrings to come to market with a focus on health tracking, rather than the previous NFC-enabled smartrings~\cite{OuraWiki}. Since Oura’s launch in 2015, they have remained a constant in the smartring space, even as competitors launched their own smartrings. In the past decade, the Oura Ring’s technology has been refined and homogenized across brands: sleep, activity, and general health are recorded by three to five sensors, and that data is interpreted by a companion phone app. The physical specifications have also been refined, with most smartrings being 2.5--5 grams, usually made with titanium, and with multiple days of battery life. Since current smartrings have similar physical specifications, their companies have to find something else to specialize in to help them stand out from the rest.

\section{Market}
Thanks to Oura’s relative longevity, their specialty is being the well-rounded smartring. Regardless of their ring’s comparative performance, they provide a consistently reliable product alongside a polished, useful phone app. One Oura competitor, the Samsung Galaxy Ring, differentiates itself by including tight software integration with their phones, most notably through finger tracking~\cite{SamsungWeb}. The Ultrahuman Ring AIR is another contender with added emphasis on how to decrease stress and improve general health through tips on their app~\cite{UltrahumanWeb}. A last competitor is the RingConn Gen 2 Air, which positions itself as a budget alternative that does not compromise on data accuracy~\cite{RingConnWeb}. All four smartring companies market themselves towards the health-conscious consumer who does not want a smartwatch, and each carves out a small niche within that market. This more minimalist market that Oura occupies functions as a middle ground between feature-complete smartwatches and simpler, 'dumb' smart wearables that focus on doing one thing well. The Pebble Index 01 is Core Devices' first foray into the smartring market, pulling from their previous experience in making 'dumb' smartwatches. The Index 01 serves as external memory for the wearer's brain by allowing the user to record voice notes and add to to-do lists~\cite{PebbleWeb}. This smartring similarly is marketed towards people who want a minimalist smartwatch alternative, but focuses more on being simple technology that keeps the user off their phone, rather than distracted. The Index 01 also has a companion app, which interprets the voice memo and creates notes or reminders accordingly. However, this app is designed to primarily work in the background, as opposed to health-focused smartring apps which maximize the health data they can give the user. Table~\ref{tab:combined_ring_comparison} goes into greater detail on the differences between each ring.

\begin{table}[htbp]
\caption{Technical and Functional Comparison of Leading Smart Rings}
\label{tab:combined_ring_comparison}
\centering
\renewcommand{\arraystretch}{1.5}
\begin{tabular}{@{}lcccc@{}}
\toprule
\textbf{Feature} & \textbf{Oura Ring 4~\cite{OuraWeb}} & \textbf{Galaxy Ring~\cite{SamsungWeb}} & \textbf{Ultrahuman Air~\cite{UltrahumanWeb}} & \textbf{RingConn Gen 2~\cite{RingConnWeb}} \\ \midrule
\textbf{Sensors}      & PPG, Temp, Accel. & Optical Bio-Signal, Temp, Accel. & PPG, Temp, 6-Axis Motion & PPG, Temp, Accel. \\
\textbf{Weight}       & 3.3--5.2\,g       & 2.3--3.0\,g        & 2.4--3.6\,g       & 2.5--4.0\,g       \\
\textbf{Battery}      & 5--8 Days         & 1 Day          & 4--6 Days         & Up to 10 Days     \\
\textbf{Charging}     & Desktop Dock      & Portable Case   & Desktop Dock      & Portable Case \\ \addlinespace[6pt]
\textbf{Durability}   & 100\,m Water Res. & 100\,m (10ATM) / IP68    & 100\,m Water Res. & IP68 Rated        \\
\textbf{Materials}    & Titanium          & Titanium                 & Titanium/Epoxy    & Titanium/Epoxy    \\ \addlinespace[6pt]
\textbf{Ecosystem}    & iOS / Android     & Android Only    & iOS / Android     & iOS / Android     \\
\textbf{Key Software} & Readiness/Stress  & Galaxy AI                & Metabolic Health  & AI Partner        \\
\textbf{Subscription} & Required & None                     & None              & None              \\ \bottomrule
\end{tabular}
\end{table}

\section{Sensors}
The Oura Ring 4 uses infrared LEDs for Photoplethysmogram (PPG), a digital temperature sensor, and an accelerometer to track health data~\cite{OuraWeb}. The infrared LEDs measure blood oxygen levels, heart rate, and heart rate variability. The Negative Temperature Coefficient (NTC) temperature sensor tracks temperature trends and variation, and the accelerometer tracks movement and activity. Using the readings from these three sensors, Oura creates a full picture of the user's health, with their mobile app tracking user readiness, sleep, activity, daytime stress and resilience, heart health, and women's health~\cite{OuraWeb}. Oura's competitors use similar sensors, with one exception: the Ultrahuman AIR uses a 6-axis motion sensor instead of an accelerometer~\cite{UltrahumanWeb}. Oura's accelerometer measures accelerations relative to the three Cartesian coordinate axis~\cite{AccelerometerVsGyro}. This measurement allows for tracking the changes in acceleration of a point, using the sensor itself as reference. Thus, an accelerometer cannot track its own position or rotation. A 6-axis motion sensor combines the accelerometer with a gyroscope which tracks changes in orientation around a reference axis, allowing for position and rotation tracking.

\section{Data Handling and Access}
Oura, and all of its competitors, use Bluetooth Low Energy (BLE) to transmit data to their phone apps~\cite{OuraWeb}. The Oura, Ultrahuman, and RingConn rings support both Android and iOS, while the Samsung ring only supports android~\cite{SamsungWeb}. The ring can store about a day's worth of health data locally, which is periodically sent over BLE to the phone app using data encryption~\cite{OuraWebFAQ}.

The smartring wearer interacts with the data from the ring in multiple ways. Primarily, the ring's data is condensed into useful information on the mobile app, with in-depth analysis on various aspects of the wearer's health. Oura splits their health insights into six sections: sleep, activity and fitness, readiness, stress, heart health, and women's health~\cite{OuraWeb}. Within each section are more detailed daily statistics, and goals for the user to hit, to incentivize being healthy. Oura's competitors have similar apps, with differently named sections but similar data. Additionally, Oura's phone app will send notifications to update the user about battery level, inactivity and activity, bedtime alerts, and general health insights~\cite{OuraWebFAQ}.

\subsection{Company Data Use}
Oura collects some data to help improve the accuracy of their rings and app~\cite{OuraDataPrivacy}. To keep customer data safe, Oura uses data encryption and anonymization or pseudonymization, depending on if the personal information would be helpful to refining their health analysis. Additionally, Oura states that they have never and will never sell the collected health data, only using it internally. The Ultrahuman shares Oura's data collection policy stance of never selling personal data~\cite{UltrahumanPrivacyPolicy}. Samsung and RingConn use encryption to keep data safe, but are less clear on their data privacy policy. Samsung and RingConn do not sell the health data, but do share that data with the same organizations they otherwise would be selling to, including Google and Meta~\cite{SamsungPrivacyPolicy, RingConnPrivacyPolicy}. While both companies claim that their data sharing is limited and for diagnostic or support purposes, they leave enough wiggle room in the exact phrasing to potentially allow for them to indirectly profit through sharing user data.

\subsection{Data Use in Research Space}
Oura uses their collected data to improve their health tracking models—for example, quantifying chronic stress. Oura measures chronic stress using a month's worth of data, which covers both daily stress reactions and long-term stress buildup~\cite{OuraChronicStress}. To create an accurate model, Oura used their collected data to create a model, and chose specific customers to self-report stress levels to help with fine-tuning. Creating accurate chronic stress models would not have been possible without using consumer data, since they would have had no way to link stress to the collected health data. 

Oura was founded as a company that specializes in sleep tracking, and they place extra emphasis on the accuracy of the smartring's accurate sleep tracking~\cite{OuraSleepTracking}. Their sleep algorithm was refined using their collected data and compared against professional sleep-tracking equipment. In a research paper published in JMIR Mhealth Uhealth, the Oura smartring's sleep tracking data accuracy is compared to an actigraphy device~\cite{AsgariMehrabadiMilad2020SToa}. The study tracked sleep data from 23 women and 22 men for 7 days, and collected information on Total Sleep Time (TST), Sleep Efficiency (SE), and Wake After Sleep Onset (WASO). Oura's smartring had notable differences in tracked information, but all deviations remained within "satisfactory range (ie, $\leqslant 30$ min for TST and WASO and <5\% for SE)", concluding that the Oura smartring is a viable alternative to medical-grade sleep tracking equipment~\cite[e20465]{AsgariMehrabadiMilad2020SToa}.

Another research paper aimed to verify Oura's step-count and energy expenditure (EE) accuracy~\cite{KristianssonEmilia2023VoOr}. The Oura Ring's EE was tested in the laboratory against an indirect calorimetry (IC), and both EE and step-count was compared against "reference monitors (three accelerometers positioned at hip, thigh, and wrist, and pedometer)" in a 14-day free-living test with 32 participants~\cite[p.~1]{KristianssonEmilia2023VoOr}. The laboratory found that the Oura Ring consistently underestimated EE compared to the IC, with greater discrepancies as physical intensity increased. The smartring fared better in the free-living test, with Oura's EE data closely correlating with the reference monitors. The smartring's step counting consistently overestimated compared to the pedometer, but matched more closely to the reference monitors' step count.

\section{Conclusion}
The Oura Ring 4 is a complete health monitoring device, which tracks six health categories using three sensors: infrared LEDs, a Negative Temperature Coefficient temperature sensor, and an accelerometer. The smartring syncs its tracked data to a phone app using Bluetooth Low Energy and encryption in transport, and that data is analyzed using fine-tuned models to represent health data. This data, especially the sleep data, has been independently verified against medical-grade equipment or other traditionally accurate alternative sensors, with generally close correlations. The Oura Ring 4 was also compared against the Samsung Galaxy Ring, the Ultrahuman Ring AIR, and the RingConn Gen 2 Air. These rings featured similar hardware, using mostly the same sensors and materials. These rings claim similar data accuracy to the Oura, but do not have research studies backing up their claims.

\bibliographystyle{IEEEtran}
\bibliography{references}

\end{document}
